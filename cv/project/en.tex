\section{Projects}

% \resumeTitle
% {HPC Application}
% {C++ OpenMP MPI}{Oct 2024}{Jul 2025}
%
% \resumeListStart
% \resumeItem{Collaborated in a team of three to parallelize serial implementation of sparse matrix-vector multiplication, merge sort, and K-means clustering using \textbf{C++} and \textbf{OpenMP} to leverage a cluster environment.}
% \resumeItem{Validated correctness of an \textbf{MPI}-based heat conduction simulation and used profiling tools (\textbf{Vampir}, \textbf{Score-P}) to analyze performance bottlenecks, addressing load imbalance and communication overhead.}
% \resumeListEnd

%\resumeSubItem{RWTH calendar mapper}
%{Automation for mapping university calendar to google calendar}

% \resumeHeading
% {Blade Defects detection}
% {\textbf{Python/OpenCV}}{May 2023}
%
% \resumeItemListStart
% %\resumeItemSmall{Modified serial version of Sparse Matrix vector multiplication, Merge sort, and K-mean clustering to utilise the full potential of a cluster with parllel programming in OpenMP.}
% \resumeItemSmall{Collaborated in a team of five to develop a program to detect defects on shaving blades (Philips) with accuracy of 96\%. \textbf{2\textsuperscript{nd} Place} at the \textbf{Fraunhofer Hackathon 2023}.}
% \resumeItemSmall{Consolidated team code intp a unified workflow, perform testing and extracted result with performance benchmarks and confusion matrix analysis.}
% \resumeItemListEnd

\resumeTitle{Web Portfolio}{Next.js Prisma.js Vercel}{Jan 2023}{Present}
\resumeListStart
\resumeItem{Developed and deployed a personal portfolio website using \textbf{Next.js} and \textbf{Prisma}, showcasing projects and skills; hosted on \textbf{Vercel}.}
\resumeListEnd

\resumeTitle{Secret santa}{Next.js}{Nov 2022}{Dec 2022}
\resumeListStart
\resumeItem{Built and deployed a web app using \textbf{Next.js} to automate Secret Santa assignments with fair, non-repeating pairings.}
\resumeListEnd

\resumeTitle
{Asclepius — ASL Translator}
{Next.js / Python / TensorFlow / OpenCV}{Mar 2022}{Jun 2022}
\resumeListStart
\resumeItem{Led a team of five to build a full-stack application for real-time American Sign Language (ASL) translation; \textbf{1\textsuperscript{st} Place} at the \textbf{Microsoft APAC AI for Accessibility Hackathon 2022}.}
\resumeItem{Developed and trained the gesture recognition model using \textbf{TensorFlow} and \textbf{OpenCV}, enabling accurate ASL-to-text conversion from webcam input.}
\resumeItem{Built a \textbf{Next.js} frontend to capture user gestures and send image data to a Python backend for live inference and text output.}
\resumeListEnd


% \resumeTitle
% {SPACE AC}{Software Engineer}{Oct 2020}{Mar 2022}
%
% \resumeListStart
%   \resumeItem{
%     \resumeSmallTitle{SPOROS}{Arduino C Python Qt5}{Nov 2020}{Jul 2021}
%   \resumeListStart
%     \resumeItem{Led end-to-end software development using \textbf{Arduino (C)} for two autorotating payloads and a CanSat relay system, and \textbf{Python/Qt5} for the ground station with real-time data visualization.}
%     \resumeItem{Designed custom communication protocols enabling mid-air telemetry relay; secured \textbf{3\textsuperscript{rd} place} in the \textbf{Annual CanSat Competition 2021}.}
%   \resumeListEnd}
%
%   \resumeItem{
%   \resumeSmallTitle{Passenger Balloon}{Arduino C Python}{Oct 2020}{Mar 2022}
%   \resumeListStart
%   \resumeItem{Contributed to \textbf{three high-altitude balloon missions}, each deploying a CubeSat payload using \textbf{Arduino (C)} and \textbf{Raspberry Pi (Python)} for autonomous image capture for atmospheric sensing and aerial imaging, reaching altitudes up to 35 km.}
%   \resumeListEnd
%   }
% \resumeItem{
% \resumeSmallTitle{Mentoring}{Arduino C Python}{Oct 2021}{Mar 2022}
%         \resumeListStart
%         \resumeItem{Designed and delivered a structured training program for new team members, covering programming fundamentals, project workflow, and hands-on development with the team's tech stack.}
%         \resumeListEnd
%  }
% \resumeListEnd
