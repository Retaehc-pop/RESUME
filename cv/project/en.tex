\section{Projects}

% \resumeTitle
% {HPC Application}
% {C++ OpenMP MPI}{Oct 2024}{Jul 2025}
%
% \resumeListStart
% \resumeItem{Collaborated in a team of three to parallelize serial implementation of sparse matrix-vector multiplication, merge sort, and K-means clustering using \textbf{C++} and \textbf{OpenMP} to leverage a cluster environment.}
% \resumeItem{Validated correctness of an \textbf{MPI}-based heat conduction simulation and used profiling tools (\textbf{Vampir}, \textbf{Score-P}) to analyze performance bottlenecks, addressing load imbalance and communication overhead.}
% \resumeListEnd


\resumeTitle
{SPOS — Embedded OS on ATmega664}{C AVR}{Apr 2024}{Jul 2024}
\resumeListStart
\resumeItem{Built a custom OS on the \textbf{ATmega664} using \textbf{C} and \textbf{AVR-GCC}, featuring interrupt handling, critical sections, task scheduling, and dynamic memory management across internal SRAM and external \textbf{23LC1024} RAM.}
\resumeItem{Developed user-facing features including LED matrix output, joystick input, and a real-time \textbf{Snake game} to showcase OS functionality.}
\resumeListEnd

% \resumeTitle
% {Asclepius, ASL translator}
% {Python OpenCV Tensorflow}{Mar 2022}{Jun 2022}
% \resumeListStart
% \resumeItem{Led a team of five in developing a machine learning–based translator for American Sign Language (ASL) to text, using real-time video processing and gesture recognition. \textbf{1\textsuperscript{st} Place} winner at the \textbf{Microsoft APAC AI for Accessibility Hackathon 2022}.}
% \resumeItem{Oversaw project planning, team coordination, and the design and training of the core deep learning model using \textbf{TensorFlow} and \textbf{OpenCV}.}
% \resumeListEnd
%
\resumeTitle
{SPACE AC}{Software Engineer}{Oct 2020}{Mar 2022}

\resumeListStart

  \resumeItem{
  \resumeSmallTitle{SPOROS}{Arduino C Python Qt5}{Nov 2020}{Jul 2021}
  \resumeListStart
    \resumeItem{Led end-to-end software development using \textbf{Arduino (C)} for two autorotating payloads and a CanSat relay system, and \textbf{Python/Qt5} for the ground station with real-time data visualization.}
    \resumeItem{Designed custom communication protocols over \textbf{XBee 3 (Zigbee 3.0)}, enabling mid-air telemetry relay between payloads, CanSat, and ground station.}
    \resumeItem{Secured \textbf{3\textsuperscript{rd} place} in the \textbf{Annual CanSat Competition 2021}.}
  \resumeListEnd
}

  \resumeItem{
  \resumeSmallTitle{AlienSat}{Arduino C Python Qt5}{Aug 2021}{Feb 2022}
  \resumeListStart
  \resumeItem{Developed software for a CanSat payload equipped with a \textbf{thermal camera} to stream raw temperature arrays in real-time for environmental analysis of PM2.5–heat correlation.}
    \resumeItem{Implemented live telemetry and thermal data visualization in the \textbf{Python/Qt5}-based ground station; handled direct payload-to-ground communication protocol design.}
  \resumeListEnd
  }

  \resumeItem{
  \resumeSmallTitle{Passenger Balloon}{Arduino C Python}{Oct 2020}{Mar 2022}
  \resumeListStart
  \resumeItem{Contributed to \textbf{three high-altitude balloon missions}, each deploying a CubeSat payload using \textbf{Arduino (C)} and \textbf{Raspberry Pi (Python)} for autonomous image capture for atmospheric sensing and aerial imaging, reaching altitudes up to 35 km.}
  \resumeListEnd
  }
\resumeItem{
\resumeSmallTitle{Mentoring}{Arduino C Python}{Oct 2021}{Mar 2022}
        \resumeListStart
        \resumeItem{Designed and delivered a structured training program for new team members, covering programming fundamentals, project workflow, and hands-on development with the team's tech stack.}
        \resumeListEnd
 }
\resumeListEnd
